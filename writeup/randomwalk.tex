\documentclass{article}
\input{~/Documents/university/preamble.tex}
\usepackage{geometry}[margin=1.0in]
\usepackage[T1]{fontenc}
\usepackage{tgpagella}
\usepackage{siunitx}
\usepackage{graphicx}
\usepackage{hyperref}
\usepackage{algpseudocode}
\hypersetup{colorlinks=true,allcolors=blue}

\title{Random Walks in $d$ Dimensions}
\author{Rudolph John Rodriguez \\ \\  PHYS 426 -- Coding Project}
\date{\today}
\begin{document}

\maketitle

\section{Modeling the Random Walk}
\subsection{Mathematical Definitions}
A one-dimensional random walk my be modeled as a vector $\mathbf{V}_\text{walk}$ of dimension $n$, where $n$ is the number of steps taken in the walk. We write $\mathbf{V}_\text{walk}$ as 

\begin{equation}
    \mathbf{V}_\text{walk} = (x_1, x_2,\cdots , x_n)
\end{equation}

where $x_i = \pm 1$. We define the position vector at step number $n$ to be
\begin{equation}
    \mathbf{S}_n = \sum_{i=1}^n x_i.
\end{equation}

In one dimension, our position vector is a scalar. We may generalize the above to $d$ dimensions, then $\mathbf{V}_\text{walk}$ becomes a vector with entries being coordinates in $d$ dimensions;
\begin{equation} 
    \mathbf{V}_\text{walk} = \{(x_{1,1}, \cdots x_{1,d}), (x_{2,1}, \cdots x_{2,d}),\cdots , (x_{n,1}, \cdots x_{n,d})\}.
\end{equation}
Our position vector becomes
\begin{equation}
    \mathbf{S}_n = \sum_{i=1}^n (x_{i,1}, \cdots, x_{i,d}).
\end{equation}

We are interested in the case where the walk returns to the origin. In our defined quantities, this means that $\mathbf{V}_\text{walk}$ of step number $n$ is such that $\mathbf{S}_n$ is the zero vector in $\mathbb{R}^d$.

\subsection{The Algorithm}

To realize a $d$ dimensional random walk we begin by defining an axis set;
\begin{equation}
    A = \{1,2,3,\cdots, d\},
\end{equation}
and a direction set;
\begin{equation}
    D = \{1,-1\}.
\end{equation}
For example, in two dimensions the element $1 \in A$ may correspond to the $x$ axis in the $2d$ plane, where $2 \in A$ may correspond to the $y$ axis. Then, $\pm 1 \in D$ correspond to a positive or negative step of displacement 1 in whichever axis is selected.

We now define $p \in [0,1]$ to be the probability to step in the positive direction, and $q = 1-p$ to be the probability to step in the negative direction.

We may now enumerate the steps in our random walk Algorithm. Note that in the below, we may adjust the size of the axis set $A$ to simulate random walks in an arbitrary number of dimensions. We may also adjust $p$ to bias the walk.

\textbf{Random walk Algorithm:}
\begin{enumerate}
    \item Initialize the position vector at $0$ steps to  $\mathbf{S}_0 = \mathbf{0}$.
    \item Take an unbiased sample $a$ from $A$.
    \item Take a sample $\delta$ from $D$, where $p$ is the probability to select $1$ and $q$ is the probability to select $-1$.
    \item Increment the index of $\mathbf{S}$ by 1 and the  $a$-th coordinate of $\mathbf{S}$ by $\delta$.
    \item Check is $\mathbf{S} = 0$. If so, return the number of steps taken so far. If not, repeat steps 2-4.
\end{enumerate}

This concludes the definition of our random walk algorithm.

\section{Experimental Details}

\section{Theory}

\section{Data Analysis}

\section{Error Analysis}

\section{Discussion}

\section{Conclusion}

\end{document}